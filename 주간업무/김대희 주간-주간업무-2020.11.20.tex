%	-------------------------------------------------------------------------------
%
%			작성		
%				2020년 
%				10월 
%				21일 
%				수
%
%
%
%
%
%
%	-------------------------------------------------------------------------------

%	\documentclass[10pt,xcolor=pdftex,dvipsnames,table]{beamer}
%	\documentclass[10pt,blue,xcolor=pdftex,dvipsnames,table,handout]{beamer}
%	\documentclass[14pt,blue,xcolor=pdftex,dvipsnames,table,handout]{beamer}
	\documentclass[aspectratio=1610,20pt,xcolor=pdftex,dvipsnames,table,handout]{beamer}
%	\documentclass[aspectratio=169,17pt,xcolor=pdftex,dvipsnames,table,handout]{beamer}
%	\documentclass[aspectratio=149,17pt,xcolor=pdftex,dvipsnames,table,handout]{beamer}
%	\documentclass[aspectratio=54,17pt,xcolor=pdftex,dvipsnames,table,handout]{beamer}
%	\documentclass[aspectratio=43,17pt,xcolor=pdftex,dvipsnames,table,handout]{beamer}
%	\documentclass[aspectratio=32,17pt,xcolor=pdftex,dvipsnames,table,handout]{beamer}

		% Font Size
		%	default font size : 11 pt
		%	8,9,10,11,12,14,17,20
		%
		% 	put frame titles 
		% 		1) 	slideatop
		%		2) 	slide centered
		%
		%	navigation bar
		% 		1)	compress
		%		2)	uncompressed
		%
		%	Color
		%		1) blue
		%		2) red
		%		3) brown
		%		4) black and white	
		%
		%	Output
		%		1)  	[default]	
		%		2)	[handout]		for PDF handouts
		%		3) 	[trans]		for PDF transparency
		%		4)	[notes=hide/show/only]

		%	Text and Math Font
		% 		1)	[sans]
		% 		2)	[sefif]
		%		3) 	[mathsans]
		%		4)	[mathserif]


		%	---------------------------------------------------------	
		%	슬라이드 크기 설정 ( 128mm X 96mm )
		%	---------------------------------------------------------	
%			\setbeamersize{text margin left=2mm}
%			\setbeamersize{text margin right=2mm}

		%	---------------------------------------------------------	
		%	슬라이드 크기 설정 ( 128mm X 96mm )
		%	---------------------------------------------------------	

%			% Format presentation size to A4
%			\usepackage[size=a4]{beamerposter}		% A4용지 크기 사용
			\geometry{paper=a5paper}
%			% Format presentation size to A4 길게
%			\geometry{paper=a4paper, landscape}

			\setbeamersize{text margin left=10mm}
			\setbeamersize{text margin right=10mm}


	%	========================================================== 	Package
		\usepackage{kotex}						% 한글 사용
		\usepackage{amssymb,amsfonts,amsmath}	% 수학 수식 사용
		\usepackage{color}					%
		\usepackage{colortbl}					%


	%		========================================================= 	note 옵션인 
	%			\setbeameroption{show only notes}
		

	%		========================================================= 	Theme

		%	---------------------------------------------------------	
		%	전체 테마
		%	---------------------------------------------------------	
		%	테마 명명의 관례 : 도시 이름
%			\usetheme{default}			%
%			\usetheme{Madrid}    		%
%			\usetheme{CambridgeUS}    	% -red, no navigation bar
%			\usetheme{Antibes}			% -blueish, tree-like navigation bar

		%	----------------- table of contents in sidebar
			\usetheme{Berkeley}		% -blueish, table of contents in sidebar
									% 개인적으로 마음에 듬

%			\usetheme{Marburg}			% - sidebar on the right
%			\usetheme{Hannover}		% 왼쪽에 마크
%			\usetheme{Berlin}			% - navigation bar in the headline
%			\usetheme{Szeged}			% - navigation bar in the headline, horizontal lines
%			\usetheme{Malmoe}			% - section/subsection in the headline

%			\usetheme{Singapore}
%			\usetheme{Amsterdam}

		%	---------------------------------------------------------	
		%	색 테마
		%	---------------------------------------------------------	
%			\usecolortheme{albatross}	% 바탕 파란
%			\usecolortheme{crane}		% 바탕 흰색
%			\usecolortheme{beetle}		% 바탕 회색
%			\usecolortheme{dove}		% 전체적으로 흰색
%			\usecolortheme{fly}		% 전체적으로 회색
%			\usecolortheme{seagull}	% 휜색
%			\usecolortheme{wolverine}	& 제목이 노란색
%			\usecolortheme{beaver}

		%	---------------------------------------------------------	
		%	Inner Color Theme 			내부 색 테마 ( 블록의 색 )
		%	---------------------------------------------------------	

%			\usecolortheme{rose}		% 흰색
%			\usecolortheme{lily}		% 색 안 칠한다
%			\usecolortheme{orchid} 	% 진하게

		%	---------------------------------------------------------	
		%	Outter Color Theme 		외부 색 테마 ( 머리말, 고리말, 사이드바 )
		%	---------------------------------------------------------	

%			\usecolortheme{whale}		% 진하다
%			\usecolortheme{dolphin}	% 중간
%			\usecolortheme{seahorse}	% 연하다

		%	---------------------------------------------------------	
		%	Font Theme 				폰트 테마
		%	---------------------------------------------------------	
%			\usfonttheme{default}		
			\usefonttheme{serif}			
%			\usefonttheme{structurebold}			
%			\usefonttheme{structureitalicserif}			
%			\usefonttheme{structuresmallcapsserif}			



		%	---------------------------------------------------------	
		%	Inner Theme 				
		%	---------------------------------------------------------	

%			\useinnertheme{default}
			\useinnertheme{circles}		% 원문자			
%			\useinnertheme{rectangles}		% 사각문자			
%			\useinnertheme{rounded}		% 깨어짐
%			\useinnertheme{inmargin}			




		%	---------------------------------------------------------	
		%	이동 단추 삭제
		%	---------------------------------------------------------	
%			\setbeamertemplate{navigation symbols}{}

		%	---------------------------------------------------------	
		%	문서 정보 표시 꼬리말 적용
		%	---------------------------------------------------------	
%			\useoutertheme{infolines}


			
	%	---------------------------------------------------------- 	배경이미지 지정
%			\pgfdeclareimage[width=\paperwidth,height=\paperheight]{bgimage}{./fig/Chrysanthemum.jpg}
%			\setbeamertemplate{background canvas}{\pgfuseimage{bgimage}}

		%	---------------------------------------------------------	
		% 	본문 글꼴색 지정
		%	---------------------------------------------------------	
%			\setbeamercolor{normal text}{fg=purple}
%			\setbeamercolor{normal text}{fg=red!80}	% 숫자는 투명도 표시


		%	---------------------------------------------------------	
		%	itemize 모양 설정
		%	---------------------------------------------------------	
%			\setbeamertemplate{items}[ball]
%			\setbeamertemplate{items}[circle]
%			\setbeamertemplate{items}[rectangle]






		\setbeamercovered{dynamic}





		% --------------------------------- 	문서 기본 사항 설정
		\setcounter{secnumdepth}{3} 		% 문단 번호 깊이
		\setcounter{tocdepth}{3} 			% 문단 번호 깊이




% ------------------------------------------------------------------------------
% Begin document (Content goes below)
% ------------------------------------------------------------------------------
	\begin{document}
	

			\title{ 주간 업무 일지 }
			\author{ 김대희 }
			\date{ 2020년 
						11월 
						20일\\ 
						48째주 
						NO.15  }


% -----------------------------------------------------------------------------
%		개정 내용
% -----------------------------------------------------------------------------
%
%		2020년 10월 5일 첫제작
%		2020.11.20 금 15판 제작					--------------------------------------->
%
%


	%	==========================================================
	%
	%	----------------------------------------------------------
		\begin{frame}[plain]
		\titlepage
		\end{frame}


		\begin{frame} [plain]{목차}
		\tableofcontents%


			\setlength{\leftmargini}{ 2em}			
			\begin{itemize}

				\item [part1] \ref{part1}	정리
%\label{part1} 	%  정리
				\item [part2] \ref{part2}	일정
%\label{part2} 	%  일정
				\item [part3] \ref{part3}	업무
%\label{part3} 	%  업무
%				\item [part4] \ref{part4}	좋은 문구
%				\item [part5] \ref{part5}	출입문			
%				\item [part6] \ref{part6}	일반사이트		
%				\item [part7] \ref{part7}	쇼핑 사이트		

			\end{itemize}


		\end{frame}



	%	========================================================== 정리
		\part{정리 }
		\frame{\partpage}

\label{part1} 	%  정리

		\begin{frame} [plain]{목차}
		\tableofcontents%
		\end{frame}
		

	%	---------------------------------------------------------- 울산지사
	%		Frame
	%	----------------------------------------------------------
		\section{울산지사}
		\begin{frame} [t,plain]
		\frametitle{울산지사}
			\begin{block} {울산지사}
			\setlength{\leftmargini}{2em}			
			\begin{itemize}
				\item 44971
				\item 울산광역시 울주군 온양읍 광청로 796-66
				\item 온양읍 운화리 384-1번지
				\item 한국도로공사 울산지사 도로안전팀
				\item 052) 701 - 6200 ASRS
				\item 052) 701 - 6271 상황실
				\item 052) 701 - 6298 도로안전팀 FAX
				\item 				\hrulefill

			\end{itemize}
			\end{block}						

		\end{frame}						
		

	%	---------------------------------------------------------- 청출
	%		Frame
	%	----------------------------------------------------------
		\section{청출}
		\begin{frame} [t,plain]
		\frametitle{청출}
			\begin{block} {청출}
			\setlength{\leftmargini}{1em}			
			\begin{itemize}
				\item 박일 \quad \hrulefill
				\item 오동재		\quad \hrulefill
				\item 심경환		\quad \hrulefill
				\item 강민재		\quad \hrulefill
				\item 김민자		\quad \hrulefill
				\item 홍성화		\quad \hrulefill

			\end{itemize}
			\end{block}			

								
		\end{frame}						
	
	%	---------------------------------------------------------- 절로
	%		Frame
	%	----------------------------------------------------------
		\section{절로}
		\begin{frame} [t,plain]
		\frametitle{절로}
			\begin{block} {절로}
			\setlength{\leftmargini}{1em}			
			\begin{itemize}
				\item 	\hrulefill
				\item 	\hrulefill
				\item 	\hrulefill
				\item 	\hrulefill
				\item 	\hrulefill
				\item 	\hrulefill
				\item 	\hrulefill
			\end{itemize}
			\end{block}			
								
		\end{frame}						


	%	---------------------------------------------------------- 초량 집
	%		Frame
	%	----------------------------------------------------------
		\section{초량 집}
		\begin{frame} [t,plain]
		\frametitle{초량 집}
			\begin{block} {초량 집}
			\setlength{\leftmargini}{1em}			
			\begin{itemize}
				\item 	\hrulefill	김 대희
				\item 	\hrulefill	안 신영
				\item 	\hrulefill	김 재민
				\item 	\hrulefill	김 재경
				\item 	\hrulefill	김 재홍
				\item 	\hrulefill	조 경연
				\item 	\hrulefill	전 삼연
			\end{itemize}
			\end{block}			
								
		\end{frame}						
			

	%	========================================================== 일정
		\part{일정}
		\frame{\partpage}
		
\label{part2} 	%  일정

		\begin{frame} [plain]{목차}
		\tableofcontents%
		\end{frame}

	%	---------------------------------------------------------- 주간
	%		Frame
	%	----------------------------------------------------------
		\section{주간 
				46째주 }

		\begin{frame} [t,plain]
		\frametitle{}
			\begin{block} {주간 46째주 }
			\setlength{\leftmargini}{1em}			
			\begin{itemize}
				\item 월	\hrulefill
				\item \hrulefill
				\item \hrulefill
				\item 화 \hrulefill
				\item \hrulefill
				\item \hrulefill
				\item 수 \hrulefill
				\item \hrulefill
				\item \hrulefill
				\item 목 \hrulefill
				\item \hrulefill
				\item \hrulefill
				\item 금 \hrulefill
				\item \hrulefill
				\item \hrulefill
				\item 토 \hrulefill
				\item \hrulefill
				\item \hrulefill
				\item 일 \hrulefill
				\item \hrulefill
				\item \hrulefill
			\end{itemize}
			\end{block}			
								
		\end{frame}						
		

	%	---------------------------------------------------------- 월 11/23 } 
	%		Frame
	%	----------------------------------------------------------
		\section{월 11/23 } 
		\begin{frame} [t,plain]
		\frametitle{}
			\begin{block} {월 11/23 } 
			\setlength{\leftmargini}{3em}			
			\begin{itemize}
				\item [06-07]	\hrulefill		  
				\item [07-08]	\hrulefill
				\item [08-09]	\hrulefill
				\item [09-10]	\hrulefill
				\item [10-11]	\hrulefill
				\item [11-12]	\hrulefill
				\item [12-01]	\hrulefill
				\item [01-02]	\hrulefill
				\item [02-03]	\hrulefill
				\item [03-04]	\hrulefill
				\item [04-05]	\hrulefill
				\item [05-06]	\hrulefill
				\item [06-07]	\hrulefill
				\item [07-08]	\hrulefill 반야선원 유식
				\item [08-09]	\hrulefill
				\item [09-10]	\hrulefill
				\item [10-11]	\hrulefill
				\item [11-12]	\hrulefill
			\end{itemize}
			\end{block}			
								
		\end{frame}						


	%	---------------------------------------------------------- 화 11/24 }
	%		Frame
	%	----------------------------------------------------------
		\section{화 11/24 }
		\begin{frame} [t,plain]
		\frametitle{}
			\begin{block} {화 11/24 }
			\setlength{\leftmargini}{3em}			
			\begin{itemize}
				\item [06-07]	\hrulefill		  
				\item [07-08]	\hrulefill
				\item [08-09]	\hrulefill
				\item [09-10]	\hrulefill
				\item [10-11]	\hrulefill
				\item [11-12]	\hrulefill
				\item [12-01]	\hrulefill
				\item [01-02]	\hrulefill
				\item [02-03]	\hrulefill
				\item [03-04]	\hrulefill
				\item [04-05]	\hrulefill
				\item [05-06]	\hrulefill
				\item [06-07]	\hrulefill
				\item [07-08]	\hrulefill
				\item [08-09]	\hrulefill
				\item [09-10]	\hrulefill
				\item [10-11]	\hrulefill
				\item [11-12]	\hrulefill
			\end{itemize}
			\end{block}			
		\end{frame}						


	%	---------------------------------------------------------- 수 11/25 }
	%		Frame
	%	----------------------------------------------------------
		\section{수 11/25 }
		\begin{frame} [t,plain]
		\frametitle{}
			\begin{block} {수 11/25 }
			\setlength{\leftmargini}{3em}			
			\begin{itemize}
				\item [06-07]	\hrulefill		  
				\item [07-08]	\hrulefill
				\item [08-09]	\hrulefill
				\item [09-10]	\hrulefill
				\item [10-11]	\hrulefill
				\item [11-12]	\hrulefill
				\item [12-01]	\hrulefill
				\item [01-02]	\hrulefill
				\item [02-03]	\hrulefill
				\item [03-04]	\hrulefill
				\item [04-05]	\hrulefill
				\item [05-06]	\hrulefill
				\item [06-07]	\hrulefill
				\item [07-08]	\hrulefill
				\item [08-09]	\hrulefill
				\item [09-10]	\hrulefill
				\item [10-11]	\hrulefill
				\item [11-12]	\hrulefill
			\end{itemize}
			\end{block}			
		\end{frame}						


	%	---------------------------------------------------------- 목 11/26  }
	%		Frame
	%	----------------------------------------------------------
		\section{목 11/26  }
		\begin{frame} [t,plain]
		\frametitle{}
			\begin{block} {목 11/26  }
			\setlength{\leftmargini}{3em}			
			\begin{itemize}
				\item [06-07]	\hrulefill		  
				\item [07-08]	\hrulefill
				\item [08-09]	\hrulefill
				\item [09-10]	\hrulefill
				\item [10-11]	\hrulefill
				\item [11-12]	\hrulefill
				\item [12-01]	\hrulefill
				\item [01-02]	\hrulefill
				\item [02-03]	\hrulefill
				\item [03-04]	\hrulefill
				\item [04-05]	\hrulefill
				\item [05-06]	\hrulefill
				\item [06-07]	\hrulefill
				\item [07-08]	\hrulefill
				\item [08-09]	\hrulefill
				\item [09-10]	\hrulefill
				\item [10-11]	\hrulefill
				\item [11-12]	\hrulefill
			\end{itemize}
			\end{block}			
		\end{frame}						



	% 	---------------------------------------------------------- 금	11/27 }
	% 	Frame					
	% 	----------------------------------------------------------					
		\section{금	11/27 }
		\begin{frame} [t,plain]					
		\frametitle{}					
			\begin{block} {금	11/27 }
			\setlength{\leftmargini}{3em}					
			\begin{itemize}					
				\item [06-07]	\hrulefill		  
				\item [07-08]	\hrulefill
				\item [08-09]	\hrulefill
				\item [09-10]	\hrulefill
				\item [10-11]	\hrulefill
				\item [11-12]	\hrulefill
				\item [12-01]	\hrulefill
				\item [01-02]	\hrulefill
				\item [02-03]	\hrulefill
				\item [03-04]	\hrulefill
				\item [04-05]	\hrulefill
				\item [05-06]	\hrulefill
				\item [06-07]	\hrulefill
				\item [07-08]	\hrulefill
				\item [08-09]	\hrulefill
				\item [09-10]	\hrulefill
				\item [10-11]	\hrulefill
				\item [11-12]	\hrulefill
			\end{itemize}					
			\end{block}					
		\end{frame}					


	% ---------------------------------------------------------- 토 11/28 }
	% Frame			
	% ----------------------------------------------------------			
		\section{토 11/28 }
		\begin{frame} [t,plain]		
		\frametitle{}		
			\begin{block} {토 11/28 }
			\setlength{\leftmargini}{3em}	
			\begin{itemize}	
				\item [06-07]	\hrulefill		  
				\item [07-08]	\hrulefill
				\item [08-09]	\hrulefill
				\item [09-10]	\hrulefill
				\item [10-11]	\hrulefill
				\item [11-12]	\hrulefill
				\item [12-01]	\hrulefill
				\item [01-02]	\hrulefill
				\item [02-03]	\hrulefill
				\item [03-04]	\hrulefill
				\item [04-05]	\hrulefill
				\item [05-06]	\hrulefill
				\item [06-07]	\hrulefill
				\item [07-08]	\hrulefill
				\item [08-09]	\hrulefill
				\item [09-10]	\hrulefill
				\item [10-11]	\hrulefill
				\item [11-12]	\hrulefill
			\end{itemize}	
			\end{block}	
		\end{frame}		

	% ---------------------------------------------------------- 일	11/29 }
	% Frame			
	% ----------------------------------------------------------			
		\section{일	11/29 }		
		\begin{frame} [t,plain]		
		\frametitle{}		
			\begin{block} {일	11/29 }
			\setlength{\leftmargini}{3em}	
			\begin{itemize}	
				\item [06-07]	\hrulefill		  
				\item [07-08]	\hrulefill
				\item [08-09]	\hrulefill
				\item [09-10]	\hrulefill
				\item [10-11]	\hrulefill
				\item [11-12]	\hrulefill
				\item [12-01]	\hrulefill
				\item [01-02]	\hrulefill
				\item [02-03]	\hrulefill
				\item [03-04]	\hrulefill
				\item [04-05]	\hrulefill
				\item [05-06]	\hrulefill
				\item [06-07]	\hrulefill
				\item [07-08]	\hrulefill
				\item [08-09]	\hrulefill
				\item [09-10]	\hrulefill
				\item [10-11]	\hrulefill
				\item [11-12]	\hrulefill

			\end{itemize}	
			\end{block}	
		\end{frame}		





	%	========================================================== 업무
		\part{업무}
		\frame{\partpage}

\label{part3} 	%  업무

		\begin{frame} [plain]{업무 목차}
		\tableofcontents%
		\end{frame}


	% ----------------------------------------------------------------------------- 금정동문회
	%
	% -----------------------------------------------------------------------------
		\section{금정동문회}
		\begin{frame} [t,plain]
		\frametitle{금정동문회}
			\begin{block} {금정동문회 051)582-7013}
			\setlength{\leftmargini}{1em}			
			\begin{itemize}
				\item 동문회 사무실 051)582-7013
				\item 포교국장 해륜스님 	\hrulefill
				\item 창화 서종현 회장	\hrulefill
				\item 김홍철 이태금 감사	\hrulefill
				\item 신기열 회장	\hrulefill
				\item 정봉호 사무총장 	\hrulefill
				\item 이채미 사무장 	\hrulefill

			\end{itemize}
			\end{block}						
		\end{frame}					


	% ----------------------------------------------------------------------------- 능인회
	%
	% -----------------------------------------------------------------------------
		\section{능인회}
		\begin{frame} [t,plain]
		\frametitle{능인회}
			\begin{block} {능인회 (반야선원) 051)701-5655 }
			\setlength{\leftmargini}{1em}			
			\begin{itemize}
				\item 허공 법사 스님 	\hrulefill 010-3732-3383
				\item 양도근 회장	\hrulefill
				\item 하승희 총무	\hrulefill
				\item 조현진 	\hrulefill
				\item 여종한	\hrulefill
				\item 전대성 감사	\hrulefill
				\item 최윤교 	\hrulefill
				\item 정일현	\hrulefill
				\item 김홍윤	\hrulefill
				\item 김원경	\hrulefill
				\item 

			\end{itemize}
			\end{block}						
		\end{frame}					

	% ----------------------------------------------------------------------------- 포교사
	%
	% -----------------------------------------------------------------------------
		\section{포교사}
		\begin{frame} [t,plain]
		\frametitle{포교사}

			\begin{block} {포교사 051)633-3011,3015}
			\setlength{\leftmargini}{1em}			
			\begin{itemize}
				\item 자재천 정분남 단장 	\hrulefill
				\item 손삼호 사무국장 \\ 010-3587-3530 \\ jeamho son353535
			\end{itemize}
			\end{block}						


			\begin{block} {포교사 청파팀 }
			\setlength{\leftmargini}{1em}			
			\begin{itemize}
				\item 팀장 본각 김영신 	\hrulefill
				\item 총무 보운 천유실 	\hrulefill
				\item 신문 월인 김필근 	\hrulefill
				\item 공성 박영찬 	\hrulefill
				\item 보광 이희규	\hrulefill

			\end{itemize}
			\end{block}						

			\begin{block} {서부지역 청파팀 법당}
			\setlength{\leftmargini}{1em}			
			\begin{itemize}
				\item 호국 백양사 2,7 	\hrulefill
				\item 호국 혜옹사 3 	\hrulefill
				\item 호국 관음사 4 	\hrulefill
				\item 모라동 대대 5,6 	\hrulefill
			\end{itemize}
			\end{block}						


		\end{frame}					


%	%	========================================================== skip
%		\begin{frame} [t,plain]
%		\end{frame}재


	% ----------------------------------------------------------------------------- 수채화
	%
	% -----------------------------------------------------------------------------
		\section{수채화}
		\begin{frame} [t,plain]
		\frametitle{수채화}
			\begin{block} {수채화}
			\setlength{\leftmargini}{1em}			
			\begin{itemize}
				\item 	\hrulefill
				\item 	\hrulefill
				\item 	\hrulefill
			\end{itemize}
			\end{block}						



			\begin{block} {포교사 시험 봉사 }
			\setlength{\leftmargini}{1em}			
			\begin{itemize}
				\item 11월19일 목 전대성,조순점,정은경	\hrulefill
				\item 12월30일 수 김둘선,김명자,조현진	\hrulefill
				\item 01월20일 수 정은경,정재황,조순점 	\hrulefill
			\end{itemize}
			\end{block}						


		\end{frame}					

	% ----------------------------------------------------------------------------- 도서관 
	%
	% -----------------------------------------------------------------------------
		\section{도서관}
		\begin{frame} [t,plain]
		\frametitle{}
			\begin{block} {도서관}
			\setlength{\leftmargini}{1em}			
			\begin{itemize}
				\item 책바다 - 국립중앙도서관
				\item 부산시립중앙도서관
				\item 밀리의서재 
			\end{itemize}
			\end{block}						
		\end{frame}					



	% ----------------------------------------------------------------------------- 풍물단
	%
	% -----------------------------------------------------------------------------
		\section{풍물단}
		\begin{frame} [t,plain]
		\frametitle{풍물단}
			\begin{block} {풍물단}
			\setlength{\leftmargini}{1em}			
			\begin{itemize}
				\item 	\hrulefill
				\item 	\hrulefill
				\item 	\hrulefill
			\end{itemize}
			\end{block}						
		\end{frame}					


	% ----------------------------------------------------------------------------- 절로
	%
	% -----------------------------------------------------------------------------
		\section{절로}
		\begin{frame} [t,plain]
		\frametitle{}
			\begin{block} {절로}
			\setlength{\leftmargini}{1em}			
			\begin{itemize}
				\item 	\hrulefill
				\item 	\hrulefill
				\item 	\hrulefill
			\end{itemize}
			\end{block}						
		\end{frame}					

	% ----------------------------------------------------------------------------- 금성고등학교
	%
	% -----------------------------------------------------------------------------
		\section{금성고등학교}
		\begin{frame} [t,plain]
		\frametitle{}
			\begin{block} {금성고등학교}
			\setlength{\leftmargini}{1em}			
			\begin{itemize}
				\item 	\hrulefill
				\item 	\hrulefill
				\item 	\hrulefill
			\end{itemize}
			\end{block}						
		\end{frame}					

	% ----------------------------------------------------------------------------- 물품 구입
	%
	% -----------------------------------------------------------------------------
		\section{물품 구입}
		\begin{frame} [t,plain]
		\frametitle{}
			\begin{block} {물품 구입}
			\setlength{\leftmargini}{1em}			
			\begin{itemize}
				\item 	\hrulefill
				\item 	\hrulefill
				\item 	\hrulefill
				\item 	\hrulefill
				\item 	\hrulefill
				\item 	\hrulefill
				\item 	\hrulefill
				\item 	\hrulefill
				\item 	\hrulefill
				\item 	\hrulefill

			\end{itemize}
			\end{block}						
		\end{frame}					








% ------------------------------------------------------------------------------
% End document
% ------------------------------------------------------------------------------





\end{document}


